\documentclass[main.tex]{subfiles}
\begin{document}
\newpage
\section{Приложения}\label{appendix}

\subsection{Тепловые карты}

\begin{figure}[H]
    \centering
    \begin{subfigure}{.5\textwidth}
        \centering
        \includegraphics[width=\linewidth]{significant/manhattan/heatmap}
        \captionsetup{width=.8\linewidth}
        \caption{расстояние Хэмминга (Манхэттенская метрика)}
        \label{fig:heatmap_manh}
    \end{subfigure}%
    \begin{subfigure}{.5\textwidth}
        \centering
        \includegraphics[width=\linewidth]{significant/impact/heatmap}
        \captionsetup{width=.8\linewidth}
        \caption{расстояние с учётом значимости}
        \label{fig:heatmap_impact}
    \end{subfigure}

    \begin{subfigure}{.5\textwidth}
        \centering
        \includegraphics[width=\linewidth]{significant/allele/heatmap}
        \captionsetup{width=.8\linewidth}
        \caption{расстояние с учётом аллельности}
        \label{fig:heatmap_allele}
    \end{subfigure}%
    \begin{subfigure}{.5\textwidth}
        \centering
        \includegraphics[width=\linewidth]{significant/impact_allele/heatmap}
        \captionsetup{width=.8\linewidth}
        \caption{расстояние с учётом значимости и аллельности}
        \label{fig:heatmap_impact_allele}
    \end{subfigure}
    \caption{Тепловые карты, построенные по матрицам расстояний (учтены только полиморфизмы с аннотациями)}
\end{figure}

\subsection{Деревья Neighbour-Joining}

\begin{figure}[H]
    \centering
    \begin{subfigure}{.5\textwidth}
        \centering
        \includegraphics[width=\linewidth]{significant/manhattan/nj}
        \captionsetup{width=.8\linewidth}
        \caption{расстояние Хэмминга (Манхэттенская метрика)}
        \label{fig:nj_manh}
    \end{subfigure}%
    \begin{subfigure}{.5\textwidth}
        \centering
        \includegraphics[width=\linewidth]{significant/impact/nj}
        \captionsetup{width=.8\linewidth}
        \caption{расстояние с учётом значимости}
        \label{fig:nj_impact}
    \end{subfigure}

    \begin{subfigure}{.5\textwidth}
        \centering
        \includegraphics[width=\linewidth]{significant/allele/nj}
        \captionsetup{width=.8\linewidth}
        \caption{расстояние с учётом аллельности}
        \label{fig:nj_allele}
    \end{subfigure}%
    \begin{subfigure}{.5\textwidth}
        \centering
        \includegraphics[width=\linewidth]{significant/impact_allele/nj}
        \captionsetup{width=.8\linewidth}
        \caption{расстояние с учётом значимости и аллельности}
        \label{fig:nj_impact_allele}
    \end{subfigure}
    \caption{Деревья, построенные метдом объединения соседей по матрицам расстояний (учтены только полиморфизмы с аннотациями)}
\end{figure}

\subsection{Распределение значений в матрице расстояний}

\begin{figure}[H]
    \centering
    \begin{subfigure}{.5\textwidth}
        \centering
        \includegraphics[width=\linewidth]{significant/manhattan/dm_histogram}
        \captionsetup{width=.8\linewidth}
        \caption{расстояние Хэмминга (Ман\-хэт\-тенская метрика)}
        \label{fig:hist_manh}
    \end{subfigure}%
    \begin{subfigure}{.5\textwidth}
        \centering
        \includegraphics[width=\linewidth]{significant/impact/dm_histogram}
        \captionsetup{width=.8\linewidth}
        \caption{расстояние с учётом значимости}
        \label{fig:hist_impact}
    \end{subfigure}

    \begin{subfigure}{.5\textwidth}
        \centering
        \includegraphics[width=\linewidth]{significant/allele/dm_histogram}
        \captionsetup{width=.8\linewidth}
        \caption{расстояние с учётом аллельности}
        \label{fig:hist_allele}
    \end{subfigure}%
    \begin{subfigure}{.5\textwidth}
        \centering
        \includegraphics[width=\linewidth]{significant/impact_allele/dm_histogram}
        \captionsetup{width=.8\linewidth}
        \caption{расстояние с учётом значимости и аллельности}
        \label{fig:hist_impact_allele}
    \end{subfigure}
    \caption{Распределение значений элементов в матрицах расстояний (учтены только полиморфизмы с аннотациями)}
\end{figure}

\subsection{t-SNE}

\begin{figure}[H]
    \centering
    \begin{subfigure}{.5\textwidth}
        \centering
        \includegraphics[width=\linewidth]{significant/tsne/2d}
        \captionsetup{width=.8\linewidth}
        \caption{данные с учётом только значимых полиморфизмов (на плоскости)}
        \label{fig:signif_tsne_2d}
    \end{subfigure}%
    \begin{subfigure}{.5\textwidth}
        \centering
        \includegraphics[width=\linewidth]{significant/tsne/3d}
        \captionsetup{width=.8\linewidth}
        \caption{данные только с учётом значимых полиморфизмов (в трёх измерениях)}
        \label{fig:signif_tsne_3d}
    \end{subfigure}

    \begin{subfigure}{.5\textwidth}
        \centering
        \includegraphics[width=\linewidth]{all/tsne/2d}
        \captionsetup{width=.8\linewidth}
        \caption{нефильтнованные данные на плоскости}
        \label{fig:all_tsne_2d}
    \end{subfigure}%
    \begin{subfigure}{.5\textwidth}
        \centering
        \includegraphics[width=\linewidth]{all/tsne/3d}
        \captionsetup{width=.8\linewidth}
        \caption{нефильтрованные данные в трёх измерениях}
        \label{fig:all_tsne_3d}
    \end{subfigure}
    \caption{Результат сокращения размерности с помощью t-SNE}
\end{figure}

\newpage

\subsection{Визуализация с помощью взвешенного графа}

Ниже на рис. \ref{fig:signif_manhattan_graph} - \ref{fig:signif_impact_allele_graph} приведены изображения графов, построенных по матрице смежности. Показана сотая доля всех рёбер с самым низким весом (4320 из 431985)

\begin{figure}[H]
    \centering \includegraphics[width=\myPictWidth]{significant/manhattan/graph}
    \caption{Граф (расстояние Хэмминга)}
    \label{fig:signif_manhattan_graph}
\end{figure}

\begin{figure}[H]
    \centering \includegraphics[width=\myPictWidth]{significant/impact/graph}
    \caption{Граф (расстояние с учётом значимости мутаций)}
    \label{fig:signif_impact_graph}
\end{figure}

\begin{figure}[H]
    \centering \includegraphics[width=\myPictWidth]{significant/allele/graph}
    \caption{Граф (расстояние с учётом аллельности)}
    \label{fig:signif_allele_graph}
\end{figure}

\begin{figure}[H]
    \centering \includegraphics[width=\myPictWidth]{significant/impact_allele/graph}
    \caption{Граф (расстояние с учётом значимости и аллельности)}
    \label{fig:signif_impact_allele_graph}
\end{figure}


\end{document}
