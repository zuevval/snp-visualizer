\documentclass[main.tex]{subfiles}
\begin{document}
\section{Введение} \label{intro}

Одним из наиболее распространённых способов анализа генетических особенностей организмов является исследование \textit{однонуклеотидных полиморфизмов} (далее ОНП) (англ. single-nucleotide polymorphism, SNP). ОНП определяется как замена одного нуклеотида в выравнивании последовательности ДНК конкретного организма относительно некоторого генома, эталонного для данной популяции, но не любая, а только такая, которая встречается у статистически значимой доли особей в популяции \cite{butler}.\\
Полиморфизмы встречаются как в генах, так и в некодирующих областях ДНК и могут быть классифицированы по значимости, то есть по степени влияния на фенотип организма. Данные об ОНП для одного индивидуума позволяют определить принадлежность к той или иной популяции, а также судить о его отличии от среднего в популяции; данные об ОНП двух организмов дают возможность составить представление об их относительном различии. Данные о полиморфизмах большой выборки индивидуумов можно использовать для филогенетического анализа. \\
Число однонуклеотидных полиморфизмов, которые могут встретиться в одном организме, зачастую превышает несколько тысяч, поэтому для интерпретации собранных в эксперименте данных удобно пользоваться средствами визуализации, например, с помощью методов сокращения размерности проецировать точки в двух- или трёхмерное пространство. В настоящей работе на примере данных об ОНП в геномах 938 людей, которые были взяты из открытого источника, рассматривается методик визуализации, основанных на расчёте матрицы расстояний между всеми геномами. Проведено сравнение ряда метрик, используемых для составления матрицы.

\newpage
\section{Исходные данные} \label{input}

\section{Описание методик} \label{alg_desc}

\subsection{Метрики}

Для каждого экземпляра генетического материала известно, какие виды полиморфизмов из заданного набора присутствуют и какова аллельность мутации: мутации нет (0), мутация есть в одной хромосоме (1) или в обеих (2). Были составлены матрицы попарных расстояний между векторами признаков, где каждый признак соответствует своему ОНП и принимает значенияе 0, 1 или 2; каждый вектор соответствует индивидууму.

В простейшем случае расстояние между геномами принимается равным расстоянию Хэмминга, то есть
\begin{equation}\label{eq:hamming}
    dist_h(g^{(1)}, g^{(2)}) = \sum_{i} | g^{(1)}_i - g^{(2)}_i |
\end{equation}
(расстояние между наборами ОНП $g^{(1)}$ и $g^{(2)}$ вычисляется как сумма модулей разности значений аллельностей).

Также были программно реализованы и применены к данным три другие формулы, которые учитывают особенности предметной области векторов:

\begin{itemize}
    \item Расстояние с учётом высокой значимости мутации в обеих аллелях: известно, что, если мутация произошла в обеих хромосомах, то ей следует придавать существенно большее значение, чем мутации в какой-либо одной из хромосом. Поэтому
\end{itemize}

\subsection{Методы визуализации}

В работе были использованы четыре метода визуализации: три из них основаны на матрице расстояний (тепловая карта, филогенетическое дерево, взвешенный граф); четвёртый (t-SNE) работает непосредственно с векторами признаков (значений аллельности ОНП).

\subsubsection{Тепловая карта}

Один из наиболее простых и общеупотребимых методов визуализации расстояний матрицы расстояний -- тепловая карта. Столбцы и строки проходят сортировку, в результате которой более близкие по значениям попарных расстояний столбцы оказываются ближе друг к другу; каждому значению расстояния ставится в соответствие цвет (в данном случае малым расстояниям соответствуют тёмно-красные клетки, большим -- светло-жёлтые). Результат для матриц, построенных по значимым ОНП для каждой из четырёх метрик, размещён в приложении (рис. \ref{fig:heatmap_manh} -- \ref{fig:heatmap_impact_allele}).


\subsubsection{Дерево Neighbour-Joining}

Один из распространённых способов компактного графического представления массива геномных данных заключается в построении филогенетического дерева, которое не только отражает близость точек, но и может указать, насколько давно предположительно произошла дивергенция между группами.

Метод соединения ближайших соседей работает с матрицей расстояний. Это пошаговый алгоритм, который строит дерево, начиная с листьев: на каждой итерации соединяет два соседних узла, то есть удалят их из дерева и добавляет новый узел, вычисляя расстояния от него до остальных узлов по формуле

$$ \rho_{km} = \frac{1}{2} (\rho_{im} + \rho_{jm} - \rho_{ij}) \thickspace \forall k $$

где $i$, $j$ -- соединяемые узлы, $m$ -- новый узел.\\

Алгоритм Neighbour-Joining хорошо зарекомендовал себя на практике при построении филогенетических деревьев \cite{durbin}, потому и был применён в данной работе. \\

Подобное филогенетические дерево строится в предположении того, что длины рёбер \textit{аддитивны}: расстояние между двумя узлами дерева равняется сумме длин рёбер на пути от одного узла к другому. Можно заметить, что все четыре функции расстояния проходят <<тест на аддивность>> -- выполняется \textit{условие четырёх точек} (four-point condition): для любых четырёх геномов $ i, j, k, l $ два расстояния из трёх $ \rho_{ij} + \rho_{kl}, \rho_{ik} + \rho_{jl}, \rho_{il} + \rho_{jk} $ не меньше, чем третье (это следует из неравенства треугольника).

\subsubsection{Взвешенный граф}


\subsubsection{t-SNE}



\section{Проведённые эксперименты}\label{experiments}

\section{Результаты}

\section{Выводы}

\section{Заключение} %  TODO объединить выводы и заключение?

\end{document}
