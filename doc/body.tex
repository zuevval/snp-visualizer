\documentclass[main.tex]{subfiles}
\begin{document}
\section{Введение} \label{intro}

Одним из наиболее распространённых способов анализа генетических особенностей организмов является исследование \textit{однонуклеотидных полиморфизмов} (далее ОНП) (англ. single-nucleotide polymorphism, SNP). ОНП определяется как замена одного нуклеотида в выравнивании последовательности ДНК конкретного организма относительно некоторого генома, эталонного для данной популяции, но не любая, а только такая, которая встречается у статистически значимой доли особей в популяции \cite{butler}.\\
Полиморфизмы встречаются как в генах, так и в некодирующих областях ДНК и могут быть классифицированы по значимости, то есть по степени влияния на фенотип организма. Данные об ОНП для одного индивидуума позволяют определить принадлежность к той или иной популяции, а также судить о его отличии от среднего в популяции; данные об ОНП двух организмов дают возможность составить представление об их относительном различии. Данные о полиморфизмах большой выборки индивидуумов можно использовать для филогенетического анализа. \\
Число однонуклеотидных полиморфизмов, которые могут встретиться в одном организме, зачастую превышает несколько тысяч, поэтому для интерпретации собранных в эксперименте данных удобно пользоваться средствами визуализации, например, с помощью методов сокращения размерности проецировать точки в двух- или трёхмерное пространство. В настоящей работе на примере данных об ОНП в геномах 938 людей, которые были взяты из открытого источника, рассматривается методик визуализации, основанных на расчёте матрицы расстояний между всеми геномами. Проведено сравнение ряда метрик, используемых для составления матрицы.

\newpage
\section{Исходные данные} \label{input}

\section{Описание алгоритмов} \label{alg_desc}

\section{Проведённые эксперименты}\label{experiments}

\section{Результаты}

\section{Выводы}

\section{Заключение} %  TODO объединить выводы и заключение?
	
\end{document}
